%%%%%%%%%%%%%%%%%
% This is an example CV created using altacv.cls (v1.1, 21 November 2016) written by
% LianTze Lim (liantze@gmail.com), based on the 
% Cv created by BusinessInsider at http://www.businessinsider.my/a-sample-resume-for-marissa-mayer-2016-7/?r=US&IR=T
% 
%% It may be distributed and/or modified under the
%% conditions of the LaTeX Project Public License, either version 1.3
%% of this license or (at your option) any later version.
%% The latest version of this license is in
%%    http://www.latex-project.org/lppl.txt
%% and version 1.3 or later is part of all distributions of LaTeX
%% version 2003/12/01 or later.
%%%%%%%%%%%%%%%%

%% If you want to use \orcid or the
%% academicons icons, add "academicons"
%% to the \documentclass options. 
%% Then compile with XeLaTeX or LuaLaTeX.
% \documentclass[10pt,a4paper,academicons]{altacv}
\documentclass[10pt,a4paper]{altacv}

%% AltaCV uses the fontawesome and academicon fonts
%% and packages. 
%% See texdoc.net/pkg/fontawecome and http://texdoc.net/pkg/academicons for full list of symbols.
%% When using the "academicons" option,
%% Compile with LuaLaTeX for best results. If you
%% want to use XeLaTeX, you may need to install
%% Academicons.ttf in your operating system's font %% folder.


% Change the page layout if you need to
\geometry{left=1cm,right=10cm,marginparwidth=7cm,marginparsep=1cm,top=1cm,bottom=1cm}

% Change the font if you want to.

% If using pdflatex:
\usepackage[utf8]{inputenc}
\usepackage[T1]{fontenc}
\usepackage[default]{lato}

% If using xelatex or lualatex:
% \setmainfont{Lato}

% Change the colours if you want to
\definecolor{VividPurple}{HTML}{2E64FE}
\definecolor{SlateGrey}{HTML}{2E2E2E}
\definecolor{LightGrey}{HTML}{666666}
\colorlet{heading}{VividPurple}
\colorlet{accent}{VividPurple}
\colorlet{emphasis}{SlateGrey}
\colorlet{body}{LightGrey}


% Change the bullets for itemize and rating marker
% for \cvskill if you want to
\renewcommand{\itemmarker}{{\small\textbullet}}
\renewcommand{\ratingmarker}{\faCircle}

%% sample.bib contains your publications
\addbibresource{sample.bib}

\begin{document}
\name{MANUEL ARROYAVE}
  \tagline{  }
% Cropped to square from https://en.wikipedia.org/wiki/Marissa_Mayer#/media/File:Marissa_Mayer_May_2014_(cropped).jpg, CC-BY 2.0
\photo{3cm}{110940.jpg}
\personalinfo{%
  % Not all of these are required!
  % You can add your own with \printinfo{symbol}{detail}
  \email{matheos.mc2@gmail.com}
  %\mailaddress{ 67 street \#53 - 108 }
  % \phone{+33666312478}
 
  
  \phone{+57 3192540509}   \location{Medellín, Colombia}     %\quad{Age: 30}
  
  %\github{https://github.com/matheos} % I'm just making this up though.
  %\orcid{orcid.org/0000-0000-0000-0000} % Obviously making this up too. If you want to use this field (and also other academicons symbols), add "academicons" option to \documentclass{altacv}
}

%% Make the header extend all the way to the right, if you want. Extend the right margin by 8cm (=6.8cm marginparwidth + 1.2cm marginparsep)
\begin{adjustwidth}{}{-8cm}
\makecvheader
\end{adjustwidth}

%% Provide the file name containing the sidebar contents as an optional parameter to \cvsection.
%% You can always just use \marginpar{...} if you do
%% not need to align the top of the contents to any
%% \cvsection title in the "main" bar.
\cvsection[page1sidebar]{PROFESSIONAL SUMARY}

\cvevent{FULL TIME RESEARCHER}{DUNE Collaboration Research}{August 2019 -- Present}{EIA}
Coordinate design and optimization of digital circuits and high performance for Detector electronics for Acquiring PHotons from Neutrinos (DAPHNE), at DUNE colaboration.\\
Experience:
\begin{itemize}
\item Search for industry partners to develop hardware manufacture capacities.
\item Support the design of the hardware and the control scheme of the board.
\item Director of the hardware programming group of the warm electronics.
\item Study of the analog impedance of the analog inputs.
\item Analisys of the signal due to wavelets algorithms.
\item Power optimization at gateware level.
\end{itemize}
\divider
\cvevent{HARDWARE DEVELOPER AND SUPPLIER}{}{} {}
Supplier of hardware for different instrumentation groups at University of Antioquia.\\
\divider
\cvevent{}{Kerr Ellipsometer}{November 2018 -- May 2019} {GES, U. de A.}
Construction of each one of the instrumentation modules of a Kerr Ellipsometer in order to measure magnetic properties of thin films. (Dielectric Tensor Calculation).\\

Each one of the following items where developed:
\begin{itemize}
\item DAQ (Spartan 3 FPGA) Low distortion 3MSPS.
\item Helmholtz coils (2KOe, 6A).
\item AC power amplifier for fast switch of magnetic field (200W). 
\item Micrometric rotatory stages (5 grad secs). \item Photodiode preamplification (FDS 100 Variable Gain). 
\end{itemize}
\divider
\cvevent{}{Diverse Tribology Equipment}{April 2018 -- December 2018}{CIDEMAT}
Super slow control industrial steppers for tribology measurements: 
\begin{itemize}
\item Rockwell's measurements due to microstepers, microprocessor programming (Raspberry Pi and Arduino)
\item Vicker's due to the conditioning of microscope and mechanical timming for the motor control.
\item Calotester measurements for thin films, with arduino control.
\end{itemize}

\newpage

\cvsection[page2sidebar]{PROFESSIONAL SUMARY}

\cvevent{}{Kerr Magnetometer SIU}{November 2017 -- April 2018}{SIU, Medellín}
Data acquisition and automatization equipment (3MSPS) using Spartan 3 FPGA. Xantrex XFR150-18,  photodiodes and 1T electromagnet automatization.
\begin{itemize}
\item DAQ (Spartan 3) 
\item CORDIC II for DAQ signal.
\item LabView.
\end{itemize}
\divider
\cvevent{}{Kerr Magnetometer U. de A.}{May 2016 -- October 2018}{ Medellín}
Low cost hardware design for Kerr measurements. Implemented with a raspberry Pi. High $\Delta I / I $ signal. Very low cost.
\begin{itemize}
\end{itemize}
\divider


\cvevent{}{Physics Labs DAQ boxes}{August 2010 -- December 2011}{GES, U. de A.}
Design of DAQ systems for the physics and engineering laboratories. Arduino core.

% \cvsection{QUALITIES}
% % Adapted from @Jake's answer from http://tex.stackexchange.com/a/82729/226
% % \wheelchart{outer radius}{inner radius}{
% % comma-separated list of value/text width/color/detail}
%  \wheelchart{2cm}{1cm}{%
%   10/13em/accent!30/Organisé, 
%   25/9em/accent!60/Ponctuel,
%   5/12em/accent!10/Sociable, 
%   20/12em/accent!40/Courageux,
%   5/8em/accent!20/Capacité d'adaptation,
%   30/9em/accent/Dynamique,
%   5/8em/accent!20/Curieux}

\end{document}
